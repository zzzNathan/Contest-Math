\documentclass{article}
\usepackage{math_template}

\author{Jonathan Kasongo}
\title{STEP 3 2014 --- P2}

\begin{document}
\maketitle

\begin{problem}{STEP 3 2014 --- P2}
\begin{enumerate}
\item[(i)] Show, by means of the substitution \( u = \cosh x \), that
\[
\int \frac{\sinh x}{\cosh 2x} \, \text{d}x = \frac{1}{2\sqrt{2}} \ln \left| \frac{\sqrt{2} \cosh x - 1}{\sqrt{2} \cosh x + 1} \right| + C.
\]

\item[(ii)] Use a similar substitution to find an expression for
\[
\int \frac{\cosh x}{\cosh 2x} \, \text{d}x.
\]

\item[(iii)] Using parts (i) and (ii) above, show that
\[
\int_0^1 \frac{1}{1 + u^4} \, \text{d}u = \frac{\pi + 2 \ln\left(\sqrt{2} + 1\right)}{4 \sqrt{2}}.
\]
\end{enumerate}
\end{problem}

\begin{solution}{Solution}
Parts (i) and (ii) will be omitted here since these problems are just
straightforward $u$-subs. The real challenge here is part (iii), here was
my solution. We will call the left hand side $I$.\\

After a lot of attempts I couldn't see a clear substitution or algebraic
method that yielded the answer. But after studying the right hand side
of that equality, I guessed that the integral is probably just some linear
combination of the 2 previous integrals. This thought motivated the
following claim.\\

\textit{Claim: For some choice of real $p, q$ and  $k \neq 0$,}
$$
I = \int_p^q \frac{a \sinh x + b \cosh x}{\cosh 2x} \text{d}x
$$


\textit{Proof:} Observe that right hand side becomes

\[
\begin{aligned}
I &= \int_p^q \frac{a \sinh x + b \cosh x}{\cosh 2x} \text{d}x \\
&= \int_p^q \frac{a (e^x - e^{-x}) + b (e^x + e^{-x})}{e^{2x} + e^{-2x}}} \text{d}x \\
&= \int_p^q \frac{a(e^{3x} - e^x) + b(e^{3x} + e^x)}{e^{4x} + 1} \text{d}x \\
\end{aligned}
\]

Put $u = e^x$, $\text{d}u = e^x \text{d}x$.

\[
\begin{aligned}
I &= \int_{e^p}^{e^q} \frac{(au^3 - au) + (bu^3 + bu)}{u^4 + 1} \frac{\text{d}u}{u} \\
\end{aligned}
\]

Now if we pick $a=-b$ we are left with

$$
I = 2b \int_{e^p}^{e^q} \frac{1}{u^4 + 1} \text{d}u
$$

Now clearly if we use $b = \frac{1}{2}, a = \frac{-1}{2}, q = 0,
p \rightarrow -\infty$ the right hand side becomes $I$. \\

Now the rest of the problem is just computation, it is left as an excercise
to the reader (I'm too lazy to do it.)
\end{solution}

\end{document}
