\documentclass[11pt]{article}
\usepackage{math_template}

\title{A Solution to OC722}
\author{Jonathan Kasongo}

\begin{document}
\maketitle

\noindent
\textit{
Problem statement: Let $p$ and $q$ be distinct prime numbers. Given an infinite decreasing
arithmetic sequence in which each of the numbers $p^{23}, p^{24}, q^{23}$ and $q^{24}$ occurs,
prove that the numbers $p$ and $q$ are sure to occur in this sequence.\\
}
\vspace{0.2cm}

\noindent
WLOG assume that $p > q$. Let the common difference of consecutive terms in the sequence be $d$. \\

\noindent
{\sffamily Claim: $d \leq q-1$} \\

\noindent
For some $\lambda_1, \lambda_2, \lambda_3 \in \mathbb{Z}$,

\begin{align*}
p^{24} - p^{23} &= \lambda_1 d \\
q^{24} - q^{23} &= \lambda_2 d \\
p^{24} - q^{24} &= \lambda_3 d
\end{align*}

\noindent
We claim that $d$ has to be coprime to both $p$ and $q$. Assume that $d$ divides $p$. Then reducing the third equation modulo $p$ would imply that $q^{24} \equiv 0 \ (\text{mod} \ p)$ which is
impossible since $p$ and $q$ are distinct prime numbers. By using the symmetry of the first 2 equations this also proves that $d$ is coprime to $q$. \\

\noindent
Now consider $q^{24} - q^{23} = q^{23}(q-1) = \lambda_2 d$. Since $d$ is coprime with $q$, we must have $d \mid (q-1)$. So $q-1 \geq d \ \blacksquare$. \\

\noindent
Now to prove that $p$ and $q$ are in the sequence it suffices to prove that there exists some $k_1, k_2 \in \mathbb{Z}$ so that,

\begin{align*}
p^{24} - p &= k_1 d \\
q^{24} - q &= k_2 d \\
\end{align*}

\noindent
Equivalently we will be done if we can show that,

\begin{align*}
p^{24} &\equiv p \ (\text{mod} \ d) \\
q^{24} &\equiv q \ (\text{mod} \ d) \\
\end{align*}

\noindent
It then suffices to prove that $p^{23} \equiv q^{23} \equiv 1 \ (\text{mod} \ d)$.

\noindent
However we know that

\begin{align*}
p^{24} - p^{23} &= \lambda_1 d \\
q^{24} - q^{23} &= \lambda_2 d \\
\end{align*}

\noindent
Once we reduce modulo $d$ on both of these equations and rearrange we end up seeing that $p \equiv q \equiv 1 \ (\text{mod} \ d)$ from which the target result follows immediately.

\end{document}
