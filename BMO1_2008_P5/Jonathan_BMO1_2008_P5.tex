\documentclass{article}
\usepackage{math_template}

\author{Jonathan Kasongo}
\title{British Math Olympiad Round 1 2008 --- P5/6}

\begin{document}
\maketitle

\begin{problem}{British Math Olympiad Round 1 2008 --- P5/6}
Determine the sequences $a_0, a_1, a_2, ...$ which satisfy all of the
following conditions:

\begin{itemize}
\item $a_{n+1} = 2 a_n^2 - 1$ for every integer $n \geq 0$,
\item $a_0$ is a rational number and
\item $a_i = a_j$ for some $i,j$ with $i \neq j$.
\end{itemize}
\end{problem}

\begin{solution}{Solution}
Notice that if $a_0 > 1$ then the sequence is strictly increasing because
$a_{n+1} = 2a_n^2 - 1 > a_n \iff (2a_n + 1)(a_n - 1) > 0 \iff
a_n > 1 \ \text{or} \ a_n < -\frac{1}{2}$.
Since $a_0$ is rational, and the set of rational numbers is closed under
multiplication and subtraction, we conclude that $a_n$ is always rational.
Write $a_n = \frac{p}{q}$ for integers $|q| \geq |p|$ and
$\gcd(p, q) = 1$ and $q \neq 0$. If $|q| = |p|$ then $a_n = \pm 1$ and it
can easily be
checked that the only 2 sequences that work are $-1, 1, 1, 1,...$ and
$1, 1, 1, ...$. Now if $|q| > |p|$ then
notice that $a_{n+1} = \frac{2p^2 - q^2}{q^2}$. Because $p^2$ and $q^2$ are
coprime the fraction can only simplify if $q^2$ has a factor of 2 and
$a_{n+1} = \frac{p^2 - q^2 / 2}{q^2 / 2}$. Now observe that if
$|q| > 2$ then the denominators of each of the terms are strictly
increasing
since $q^2 > \frac{q^2}{2} > |q| \iff q^2 - 2|q| = |q|(|q|-2) > 0$ which
is clearly true when $|q|>2$. But if the denominators are strictly
increasing then there can never be 2 different numbers that are the same
in the sequence.\\

Thus $|q| \leq 2$. That means that $a_0$ is one of $0, \pm \frac{1}{2}$.
Now one can easily check each case and find that the only sequences here
are $0, -1, 1, 1, 1, ...$ and $\frac{1}{2}, -\frac{1}{2}, -\frac{1}{2}, -\frac{1}{2},...$
and $-\frac{1}{2}, - \frac{1}{2}, -\frac{1}{2}, ...$. $\blacksquare$\\

\textit{Remark: I got this idea from the field of combinatorics actually,
once I was confident that $a_0 = 0, \pm 1, \pm \frac{1}{2}$ and tested
other values of $a_0$ I noticed that the denominator increased very
quickly and that this could probably be used as a monovariant. I spent
around 3 hours thinking about this problem.}
\end{solution}

\end{document}
