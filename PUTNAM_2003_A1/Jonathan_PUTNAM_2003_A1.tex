\documentclass{article}
\usepackage{math_template}

\author{Jonathan Kasongo}
\title{Putnam 2003 --- A1}

\begin{document}
\maketitle

\begin{problem}{Putnam 2003 --- A1}
Let $n$ be a fixed positive integer. How many ways are there to
write $n$ as a sum of positive integers,

$$
n = a_1 + a_2 + \cdots + a_k
$$

with $k$ an arbitrary positive integer and
$a_1 \leq a_2 \leq \cdots \leq a_k \leq a_1 + 1$? For example, with $n=4$
there are four ways: 4, 2+2, 1+1+2, 1+1+1+1.
\end{problem}

\begin{solution}{Solution}
We say that we have represented $n$ \textit{nicely} if we have written
$n = a_1 + a_2 + \cdots + a_k$, with $a_1, a_2, ...,a_k$ satisfying the
conditions given in the problem statement.
We claim that for a fixed positive integer $n$, there are $n$ ways to
represent $n$ in the desired form. We can actually show a clever way to
prove this fact inductively. \\

The base case $n=1$ holds trivially. Now assume the claim holds for some
$n=m$. There are $m$ ways to represent $m$ nicely. Now in each of these
representations take $a_1$ and replace it with $a_1 + 1$. Every
$(a_r)_{r=1}^k$ is still either $a_1$ or $a_1 + 1$ so this is a way to
nicely represent $m+1$. The only representation that cannot be generated like this is

$$m+1 = \underbrace{1+1+\cdots+1}_{m + 1 \; \text{repititions}}$$

because this is the only representation, where you cannot take a different
nicely represented number and replace $a_1$ with $a_1 + 1$, since that
would imply that $a_1 + 1 = 1 \iff a_1 = 0$ which is impossible.\\

So for the number $m+1$ we have shown that there are $m+1$ ways to
represent it nicely, and now since we have the truth of $n=1$ and the
truth of the $n=m$ statement implies the truth of the $n=m+1$ statement,
the desired result is true for all positive integers $n$. $\blacksquare$
\end{solution}

\end{document}
