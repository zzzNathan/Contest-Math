\documentclass{article}
\usepackage{math_template}

\author{Jonathan Kasongo}
\title{Canadian Junior Mathematical Olympiad 2023 --- P1/5}

\begin{document}
\maketitle

\begin{problem}{Canadian Junior Mathematical Olympiad 2023 --- P1/5}
Let $a$ and $b$ be non-negative integers. Consider a sequence $s_1, s_2,
s_3, ...$ such that $s_1 = a, s_2 = b$ and $s_{i+1} = |s_i - s_{i-1}|$
for $i \geq 2$. Prove that there is some $i$ for which $s_i = 0$.
\end{problem}

\begin{solution}{Solution}
Suppose, for the sake of contradiction, that $s_i \neq 0$. Notice that
$\max (s_i, s_{i-1}) = \min (s_i, s_{i-1}) + s_{i+1}$ and since none of
the terms are ever 0, we have $s_{i+1} < \min (s_i, s_{i-1})$. That
means that the sequence $(s_r)_{r=1}$ contains a decreasing
non-contigous subsequence $s_{\sigma (1)} > s_{\sigma (2)} > ...$. This
means there must eventually exist some $i$ such that $s_{\sigma (i)} = 0$.
$\blacksquare$
\end{solution}

\end{document}
