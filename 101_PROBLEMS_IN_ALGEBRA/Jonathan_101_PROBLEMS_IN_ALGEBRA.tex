\documentclass{article}
\usepackage{math_template}

\author{Jonathan Kasongo}
\title{Solving Problems from 101 Problems in Algebra --- T Andreescu \& Z Feng}

\begin{document}
\maketitle

Recently I have been wanting to improve my algebra so I have been
working through \textit{101 Problems in Algebra} by
\textit{T Andreescu \& Z Feng}.

\section{Introductory Problems}

\begin{problem}{Problem 1}
Let $a,b$ and $c$ be real and positive parameters. Solve the equation

$$
\sqrt{a+bx} + \sqrt{b+cx} + \sqrt{c+ax} =
\sqrt{b-ax} + \sqrt{c-bx} + \sqrt{a-cx}
$$
\end{problem}

\begin{solution}{Solution 1}
If $x$ is a solution then that equation should hold for all triplets of
parameters $a,b,c \in \mathbb{R}^{+}$. Consider the triplet
$(a,b,c)=(1,1,1)$, we must have $3\sqrt{1+x} = 3\sqrt{1-x} \iff
1+x = 1-x \iff x=0$. So $x=0$ is the only candidate for a solution. Now
we can easily plug in $x=0$ into the above equation to see that it
indeed works for all $a,b,c$. Hence $x=0$ is the only such solution.
$\blacksquare$\\

\textit{Remark: This one took me far longer than it should have!}
\end{solution}

\vspace{0.2cm}

\begin{problem}{Problem 2}
Find the general term of the sequence defined by $x_0 = 3, x_1 = 4$ and

$$
x_{n+1} = x_{n-1}^2 - nx_n
$$

for all $n \in \mathbb{N}$.
\end{problem}

\begin{solution}{Solution 2}
After a little exploration it is not hard to see that each term $x_r$ seems
to simply be $r+3$. This guess is easily proved since
$x_{n-1}^2 - nx_n = (n+2)^2 - n(n+3) = (n^2 + 4n + 4) - (n^2 + 3n) = n+4 =
x_{n+1}$. $\blacksquare$
\end{solution}

\vspace{0.2cm}

\begin{problem}{Problem 3}
Let \( x_1, x_2, \ldots, x_n \) be a sequence of integers such that
\begin{enumerate}
    \item[(i)] \(-1 \leq x_i \leq 2,\) for \(i = 1, 2, \ldots, n;\)
    \item[(ii)] \(x_1 + x_2 + \cdots + x_n = 19;\)
    \item[(iii)] \(x_1^2 + x_2^2 + \cdots + x_n^2 = 99.\)
\end{enumerate}

Determine the minimum and maximum possible values of
\[
x_1^3 + x_2^3 + \cdots + x_n^3.
\]
\end{problem}

\begin{solution}{Solution 3}
Let $a,b,c,d$ be the number of $(-1)$'s, 0's, 1's and 2's in the sequence,
respectively. We have $-a + c + 2d = 19$ and $a + c + 4d = 99$ by condition
(i) and (ii). We are interested in $x_1^3 + x_2^3 + \cdots + x_n^3 =
-a + c + 8d = -a + (19 + a - 2d) + 8d = 19 + 6d$, using $c = 19 + a - 2d$.
But since $2c + 6d = 118 \iff c = 59 - 3d$, from adding the first 2
equations and $c\geq 0 \iff 3d \leq 59$ but that means that
$0\leq d\leq 19$. So the maximum value is $19 + 6(19) = 133$ and the
minimum is $19 + 6(0) = 19$. $\blacksquare$ \\

\textit{Remark: I had to use a hint for this one.}
\end{solution}

\vspace{0.2cm}

\begin{problem}{Problem 4}
The function $f$, defined by

$$
f(x) = \frac{ax+b}{cx+d}
$$

where $a,b,c$ and $d$ are non-zero real numbers, has the properties
$$
f(19) = 19, \ \ f(97) = 97, \ \  f(f(x)) = x
$$

for all $x$, except $-\frac{d}{c}$.\\

Find the range of $f$.
\end{problem}

\begin{solution}{Solution 4}
Clearly for all $x \neq -d/c$, $x$ is in the image of $f$ due to the third
property.
Using the third property we know that $x = \frac{af(x)+b}{cf(x)+d}$ we can
re-arrange to find that $f(x) = \frac{b-dx}{cx-a} = \frac{ax+b}{cx+d}$ so
there must exist $\lambda \in \mathbb{R}_{\neq 0}$ such that
$cx-a = \lambda (cx+d)$ and $b-dx = \lambda (ax+b)$. From the second
equation $b = b \lambda$ so $\lambda = 1$ since $b \neq 0$ therefore
$a=-d$. Now using those 2 fixed points we know that

\[
\begin{aligned}
19a + b = 19^2 c - 19a\\
97a + b = 97^2 c - 97a\\
\end{aligned}
\]

We can subtract these 2 equations to get rid of $b$ and re-arrange to get
$a = 58c$. Now $-d/c = 58c/c = 58$. So since 58 isn't in the domain of $f$
and $f$ is an involution, 58 isn't in the image of $f$ either. So the range
is $\mathbb{R} \setminus \{ 58 \}$. $\blacksquare$ \\

\textit{Remark: I had to use a hint for this one.}
\end{solution}

\vspace{0.2cm}
\begin{problem}{Problem 5}
Prove that

$$
\frac{(a-b)^2}{8a} \leq \frac{a+b}{2} - \sqrt{ab} \leq \frac{(a-b)^2}{8b}
$$

for all $a\geq b >0$.
\end{problem}

\begin{solution}{Solution 5}
The inequality is the same as

$$
\frac{(\sqrt{a} + \sqrt{b})^2(\sqrt{a} - \sqrt{b})^2}{8a} \leq
\frac{(\sqrt{a} - \sqrt{b})^2}{2} \leq
\frac{(\sqrt{a} + \sqrt{b})^2(\sqrt{a} - \sqrt{b})^2}{8b}
$$

That is

\[
\begin{aligned}
\frac{(\sqrt{a} + \sqrt{b})^2}{8a} &\leq
\frac{1}{2} \leq
\frac{(\sqrt{a} + \sqrt{b})^2}{8b} \\
b(\sqrt{a} + \sqrt{b})^2 &\leq
4ab \leq
a(\sqrt{a} + \sqrt{b})^2 \\
\end{aligned}
\]

but since $b(\sqrt{a} + \sqrt{b})^2 \leq b (2\sqrt{a})^2 = 4ab$ and
$4ab = a(2\sqrt{b})^2 \leq a(\sqrt{a} + \sqrt{b})^2$, the problem is
finished. $\blacksquare$ \\

\textit{Remark: I had to use a hint for this one.}
\end{solution}

\vspace{0.2cm}

\begin{problem}{Problem 6}
Several (at least two) nonzero numbers are written on a board. One may
erase any two numbers, say $a$ and $b$, and then write the numbers
$a + \frac{b}{2}$ and $b - \frac{a}{2}$ instead.

Prove that the set of numbers on the board, after any number of the
preceding operations, cannot coincide with the initial set.
\end{problem}

\begin{solution}{Solution 6}
Suppose we apply the operation on $a$ and $b$. The sum of their
squares is $a^2 + b^2$. Now after you apply the operation, the sum of
squares becomes $(a^2 + ab + \frac{b^2}{4}) + (b^2 - ab + \frac{a^2}{4}) =
\frac{5}{4} (a^2 + b^2)$. So the sum of squares all the numbers on the
board will strictly increase after applying each operation. This means that
after some number of operations we can never coincide with the original set
because the sum of squares of the numbers on the board, before and after
the operations, will be different.
$\blacksquare$
\\

\textit{Remark: I first came across this technique by watching Timothy
Gowers solve invariance problems on YouTube!}
\end{solution}

\vspace{0.2cm}

\begin{problem}{Problem 7}
The polynomial
\[
1 - x + x^{2} - x^{3} + \cdots + x^{16} - x^{17}
\]
may be written in the form
\[
a_0 + a_1 y + a_2 y^2 + \cdots + a_{16} y^{16} + a_{17} y^{17},
\]
where \( y = x + 1 \) and \( a_i \)s are constants. Find \( a_2 \).
\end{problem}

\begin{solution}{Solution 7}
The polynomial in terms of $y$ is

$$
1 - (y-1) + (y-1)^2 - (y-1)^3 + \cdots + (y-1)^{16} - (y-1)^{17}
$$

and we want the coefficient of $y^2$.\\

We can use binomial expansion to
find that the coefficient is
$ a_2 = \sum_{r=2}^{17} (-1)^r \binom{r}{2} (-1)^{r-2} =
\sum_{r=2}^{17} \binom{r}{2}$. But by the Hockey stick identity this is
just $\binom{18}{3} = 816$. $\blacksquare$

\end{solution}
\vspace{0.2cm}

\begin{problem}{Problem 8}
Let $a, b$ and $c$ be distinct non-zero real numbers such that

$$
a + \frac{1}{b} = b + \frac{1}{c} = c + \frac{1}{a}
$$

Prove that $|abc| = 1$.
\end{problem}

\begin{solution}{Solution 8}
The equation is symmetric, so notice that over all cyclic rotations
$(a,b,c)$ we have $a-b = \frac{1}{c} - \frac{1}{b} = \frac{b-c}{bc}$ and
we can say that $bc(a-b) = b-c$ holds over the cyclic rotations. \\

Now observe the following

\[
\begin{aligned}
b-c &= bc(a-b) \\
&= bc\left[ ab(c-a) \right] \\
&= bc\left[ ab( ac \left[ b-c \right] ) \right] \\
\end{aligned}
\]

Now we can cancel $b-c \neq 0$ since we are told $a,b,c$ are distinct
giving $1 = (abc)^2$ and so $|abc| = 1$ follows as desired. $\blacksquare$

\end{solution}

\vspace{0.2cm}

\begin{problem}{Problem 9}
Find polynomials \( f(x), g(x), \) and \( h(x), \) if they exist, such that for all \( x, \)
\[
|f(x)| - |g(x)| + h(x) =
\begin{cases}
-1 & \text{if } x < -1 \\
3x + 2 & \text{if } -1 \leq x \leq 0 \\
-2x + 2 & \text{if } x > 0.
\end{cases}
\]
\end{problem}

\begin{solution}{Solution 9}
It is natural to guess that the left hand side can be written of the form
$a|x+1| + b|x| + cx + d$. Now if you plug in the results we have and solve
the resulting simultaenous equation you end up with
$\frac{3}{2}|x+1| - \frac{5}{2}|x| - x + \frac{1}{2}$. $\blacksquare$
\end{solution}

\vspace{0.2cm}

\begin{problem}{Problem 10}
Find all real numbers \( x \) for which
\[
\frac{8^x + 27^x}{12^x + 18^x} = \frac{7}{6}.
\]
\end{problem}

\begin{solution}{Solution 10}
Let $a=3^x$, $b=2^x$. The equation becomes

\[
\begin{aligned}
\frac{b^3 + a^3}{ab^2 + a^2 b} &= \frac{7}{6}\\
6(a^3 + b^3) &= 7ab(a+b)\\
6(a^2 - ab + b^2) &= 7ab\\
6a^2 - 13ab + 6b^2 &= 0\\
\end{aligned}
\]

We can use the quadratic formula to deduce that $a = \frac{3}{2} b \iff
\frac{a}{b} = \left( \frac{3}{2} \right)^x = \frac{3}{2}$ and so $x=1$ is
the only solution. $\blacksquare$
\end{solution}

\vspace{0.2cm}

\begin{problem}{Problem 11}
Find the least positive integer $m$ such that

$$
\binom{2n}{n}^{1/n} < m
$$

holds for all positive integers $n$.
\end{problem}

\begin{solution}{Solution 11}
We claim that the answer is $m=3$. Firstly to show that $m=3$ works we
have to show that

$$
\sum_{r=0}^n \binom{n}{r}^2 = \binom{2n}{n} < 3^n = (1+2)^n = \sum_{r=0}^n \binom{n}{r} 2^r
$$

with the first equality being due to the Vandermonde Identity, and then
since
$\binom{n}{r} < \sum_{r=0}^n \binom{n}{r} = 2^r$ the claim is true. \\

Now we just have to show that $m=2$ fails. Observe that
$\binom{2}{1} = 2 < 2$ is a contradiction so indeed $m=3$ is the smallest
such $m$. $\blacksquare$
\end{solution}
\vspace{0.2cm}

\begin{problem}{Problem 12}
Let $a,b,c,d$ and $e$ be positive integers such that

$$
a+b+c+d+e = abcde
$$

Find the maximum possible value of $\max \{ a,b,c,d,e \}$
\end{problem}

\begin{solution}{Solution 12}
Suppose without loss of generality that $a\geq b\geq c \geq d \geq e$.
We can see that

$$
a = \frac{b+c+d+e}{bcde-1}
$$

So for some small $\epsilon > 0$,

\[
\begin{aligned}
a+\epsilon &\geq \frac{b+c+d+e}{bcde} \\
&= \frac{1}{cde} + \frac{1}{bde} + \frac{1}{bce} + \frac{1}{bcd} \\
\end{aligned}
\]

So we want $b,c,d,e$ to be small since we cannot have them all equal to
1 the next best permutation is $(2,1,1,1)$ which achieves the maximum value
of $a=5$. $\blacksquare$
\end{solution}

\vspace{0.2cm}

\begin{problem}{Problem 13}
Evaluate
\[
\frac{3}{1! + 2! + 3!} + \frac{4}{2! + 3! + 4!} + \cdots + \frac{2001}{1999! + 2000! + 2001!}
\]
\end{problem}

\begin{solution}{Solution 13}
It is easier to solve the more general problem. Define

\[
\begin{aligned}
S_n &= \sum_{r=1}^{n} \frac{r+2}{r! + (r+1)! + (r+2)!} \\
&= \sum_{r=1}^n \frac{r+2}{r! \left[ 1 + (r+1) + (r+1)(r+2) \right]} \\
&= \sum_{r=1}^n \frac{r+2}{r! (r+2)^2} \\
&= \sum_{r=1}^n \frac{1}{r! (r+2)}
\end{aligned}
\]

Now when you analyse the cases with small $n$, a sensible guess is that
the required form is

$$
\frac{\frac{1}{2} (n+2)! - 1}{(n+2)!} = \frac{1}{2} - \frac{1}{(n+2)!}
$$

We prove our guess with induction. The base case is true since
$S_1 = \frac{1}{3} = \frac{1}{2} - \frac{1}{6}$. Now assume that the result
is true for some $n=k$. Indeed

\[
\begin{aligned}
S_{k+1} &= S_k + \frac{1}{(k+1)!(k+3)} \\
&= \frac{1}{2} - \frac{1}{(k+2)!} + \frac{1}{(k+1)!(k+3)} \\
&= \frac{1}{2} - \frac{1}{(k+1)!} \left[ \frac{1}{k+2} - \frac{1}{k+3} \right] \\
&= \frac{1}{2} - \frac{1}{(k+1)!} \left[ \frac{1}{(k+2)(k+3)} \right] \\
&= \frac{1}{2} - \frac{1}{(k+3)!}
\end{aligned}
\]

So the result is true by induction. So the required answer is
$S_{1999} = \frac{1}{2} - \frac{1}{2001!}$. $\blacksquare$
\end{solution}
\end{document}
