\documentclass{article}
\usepackage{math_template}

\author{Jonathan Kasongo}
\title{British Math Olympiad Round 1 2011 --- P2/6}

\begin{document}
\maketitle

\begin{problem}{British Math Olympiad Round 1 2011 --- P2/6}
Consider the numbers $1, 2, . . . , n$. Find, in terms of
$n$, the largest
integer $t$ such that these numbers can be arranged in a row so that all
consecutive terms differ by at least $t$.
\end{problem}

\begin{solution}{Solution}
The answer is $t = \floor{\frac{n}{2}}$ and we split the proof into 2
separate cases.\\

\textit{Case 1: $n$ is even.} Write $n = 2m$ for some positive integer $m$.
First we will show that $t = \floor{\frac{n}{2}} = m$ works by
construction. Look at the sequence

$$
m, 2m, m-1, 2m-1, ..., 1, m+1
$$

The differences are always alternating between $m$ and $m+1$ which are both
at least $t$ so this construction works. Now we show that any $t >
\floor{\frac{n}{2}}$ doesn't work. The number $\floor{\frac{n}{2}}$
appears in our sequence, the difference with the consecutive number is at
most $\max\left(\floor{\frac{n}{2}}-1, 2n - \floor{\frac{n}{2}}\right)=
\max(m-1, m)$ and thus any $t > \floor{\frac{n}{2}} = m$ won't work since
there will be at least one pair of consecutive terms that differ by
at most $m$. \\

\textit{Case 2: $n$ is odd.} Write $n = 2m+1$ for some non-negative integer
$m$. First we show that $t = \floor{\frac{n}{2}} = m$ works by
construction. Take the construction we made for even $n$, then add
$2m+1$ onto the end of that sequence. That last difference is $m$ and all
the other differences are $m$ or $m+1$ so this value of $t$ works. Now we
show that any $t > \floor{\frac{n}{2}} = m$ doesn't work. $m+1$ is in this
sequence and the difference with the consecutive number will be at most
$\max(m+1-1, 2m-(m+1)) = \max(m, m-1)$ and thus any
$t > \floor{\frac{n}{2}} = m$ won't work because there is at least one pair
of consecutive terms that differ by at most $m$. $\blacksquare$
\end{solution}

\end{document}
