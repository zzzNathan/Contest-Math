\documentclass{article}
\usepackage{math_template}

\author{Jonathan Kasongo}
\title{British Math Olympiad Round 1 2003 --- P3/5}

\begin{document}
\maketitle

\begin{problem}{British Math Olympiad Round 1 2003 --- P3/5}
Alice and Barbara play a game with a pack of $2n$ cards, on each of
which is written a positive integer. The pack is shuffled and the cards
laid out in a row, with the numbers facing upwards. Alice starts, and
the girls take turns to remove one card from either end of the row,
until Barbara picks up the final card. Each girl’s score is the sum of
the numbers on her chosen cards at the end of the game.

Prove that Alice can always obtain a score at least as great as
Barbara’s.
\end{problem}

\begin{solution}{Solution}
Label the numbers on the cards as $a_1, a_2, ..., a_{2n}$ respectively.

Define $P_{n} := a_1 + a_3 + \cdots + a_{2n-1}$,
$Q_n := a_2 + a_4 + \cdots + a_{2n}$. \\

Notice that Alice always has a strategy to make her score
$P_n$ or $Q_n$. She should start by removing the number $a_1$ or $a_{2n}$
respectively, then Barbara will be forced to remove a number $a_k$ where
the index $k$ is of the opposite parity to the index of the number Alice
chose, since if Alice chose $a_1$ then Barbara must choose one of
$a_2, a_{2n}$ and if Alice chose $a_{2n}$ then Barbara must choose one of
$a_1, a_{2n-1}$. Now once Barbara has made her choice $a_k$, Alice can
remove the number
on the same side that Barbara made her choice. This number $a_{k \pm 1}$
will have an index of the opposite parity to the index of the number
Barbara just chose. Now if you just keep repeating this strategy then you
can clearly remove all card numbers that have the same index parity, whilst
Barbara will remove all card numbers that have an opposite index parity to
the cards Alice chose,
proving our claim.\\

Now since Alice can simply use one of the strategies above to get a score
of $\max(P_n, Q_n) \geq \min(P_n, Q_n)$,
Alice can always obtain a score at least as great as
Barbara’s. $\blacksquare$
\end{solution}

\end{document}
