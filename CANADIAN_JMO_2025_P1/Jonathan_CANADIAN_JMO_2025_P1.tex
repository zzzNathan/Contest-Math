\documentclass{article}
\usepackage{math_template}

\author{Jonathan Kasongo}
\title{Canadian Junior Math Olympiad 2025 --- P1/5}

\begin{document}
\maketitle

\begin{problem}{Canadian Junior Math Olympiad 2025 --- P1/5}
Suppose an infinite non-constant arithmetic progression of integers contains 1 in it. Prove that
there are an infinite number of perfect cubes in this progression. (A \textit{perfect cube} is an integer
of the form $k^3$, where $k$ is an integer. For example, $-8$, 0 and 1 are perfect cubes.)
\end{problem}

\begin{solution}{Solution}
Suppose the arithmetic progression has a fixed difference $d \in
\mathbb{Z}_{\neq 0}$. Notice that if we can prove that any perfect cube
$k^3 \in \mathbb{Z} \setminus \{ -1, 0, 1\}$ exists in the sequence we
automatically get that there
are an infinite number of perfect cubes in the progression since
there must exist some non-zero integer $\lambda$ such that
$1+ \lambda d = k^3$. Then this implies that $k^3 + (k^3\lambda) d = k^6$
is also part of the progression, but since $k^3 \neq k^{3n}$ for any
integer $n \geq 2$ repeating this
process will demonstrate that there are an infinite number of perfect cubes
in the sequence.\\

Hence it suffices to show that there exists $\mu \in \mathbb{Z}_{\neq 0}$
and $m^3 \in \mathbb{Z} \setminus \{ -1, 0, 1\}$ such that
$1 + \mu d = m^3$ and so $\mu d = (m - 1)(m^2 + m + 1)$. Now notice
that we can let $m - 1 = d$ and $\mu = m^2 + m + 1$, since
the discriminant of that quadratic $\Delta = 1 - 4 = -3 < 0$ shows us that
$\mu \neq 0$. So this construction works for any $d$ except $m = d+1 \neq
-1, 0, 1 \iff d \neq -2, -1, 0$. We can rule out the possibility of $d=0$
since the progression is non-constant, $d=-1$ will make the progression
simply run through all of the perfect cubes that are less than the initial
value, and for $d=-2$ the number $(-3)^3 = -27 = 1 - 2(14)$ will occur in
the sequence so it will contain an infinite number of perfect cubes by our
arguement in the first paragraph. $\blacksquare$
\end{solution}

\end{document}
