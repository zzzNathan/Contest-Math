\documentclass{article}
\usepackage{math_template}

\author{Jonathan Kasongo}
\title{British Math Olympiad 2007 Round 1 --- P3/6}

\begin{document}
\maketitle

\begin{problem}{British Math Olympiad 2007 Round 1 --- P3/6}
The number 916238457 is an example of a nine-digit number which
contains each of the digits 1 to 9 exactly once. It also has the property
that the digits 1 to 5 occur in their natural order, while the digits 1
to 6 do not. How many such numbers are there?
\end{problem}

\begin{solution}{Solution}
There are no conditions on where to place the numbers 7,8,9 so we can
ignore them for now and just place them in the remaining 3 places, once
we have placed the numbers 1 to 6. There are $\binom{9}{6}$ ways to
select 6 spaces in our nine-digit number. The number 6 must go before the
number 5 so that we can make the digits 1 to 6 be out of their natural
order. Thus we can place that number 6 in any of the first $k \leq 5$
spaces such that this subsequence of 6 spaces looks like

$$
1,2,...,k-1,6,k,...,5
$$

clearly satisfying both of the given condiions. There are thus
$5 \times \binom{9}{6}$ ways to arrange those 6 numbers and finally we
can place the numbers 7,8,9 in the remaining 3 spaces in $3!$ ways. So
the final answer is $5 \times \binom{9}{6} \times 3! = 2520$.
$\blacksquare$
\end{solution}

\end{document}
