\documentclass{article}
\usepackage{math_template}

\author{Jonathan Kasongo}
\title{British Math Olympiad Round 2 2025 --- P1/4}

\begin{document}
\maketitle

\begin{problem}{British Math Olympiad Round 2 2024 --- P1/4}
Prove that if $n$ is a positive integer, then $\frac{1}{n}$ can be written
as a finite sum of reciprocals of different triangular numbers.
\end{problem}

\begin{solution}{Solution}
Let $t_n = \frac{n(n+1)}{2}$ denote the $n^{\text{th}}$ triangular number.
\\

\textit{Claim:} For any positive integer $n$,
$\frac{1}{t_{2n}} + \frac{1}{t_{2n + 1}} = \frac{1}{2}\frac{1}{t_n}$.\\

\textit{Proof:} The left hand side reads

\[
\frac{2}{2n(2n + 1)} + \frac{2}{(2n + 1)(2n + 2)}
= \frac{2}{2n + 1} \left( \frac{1}{2n} + \frac{1}{2n + 2} \right)
= \frac{1}{2n + 1} \left( \frac{2n + 1}{n(n+1)} \right)
= \frac{1}{n(n+1)}
\]

as desired.\\

Now using our claim we have

$$
\frac{1}{n} = \frac{1}{n+1} + \frac{1}{n(n+1)} =
\frac{1}{n+1} + \left( \frac{1}{t_{2n}} + \frac{1}{t_{2n + 1}} \right)
$$

By repeatedly substituting say $k$ times, we can show that

$$
\frac{1}{n} = \frac{1}{n+k} + \sum_{r=1}^k \left( \frac{1}{t_{2r}} +
 \frac{1}{t_{2r + 1}} \right)
$$

Now if we pick $k = t_n - n = \frac{n(n-1)}{2}$ we have completed the proof
for all $n > 1$, since all those fractions are pairwise distinct since
$t_n < t_{n+1}$. But obviously $n=1$ can be written
as a finite sum of reciprocals of different triangular numbers, since 1
is a triangular number. $\blacksquare$
\end{solution}

\end{document}
